\documentclass[10pt]{examdesign}
\usepackage[utf8]{inputenc}
\usepackage[spanish]{babel}
\selectlanguage{spanish}
\usepackage{listings}
\usepackage[margin=1in]{geometry}


\lstset{
  language=Python,
  frame=single,
  basicstyle=\sffamily,
  showstringspaces=false
  %basicstyle=\ttfamily,
}

\NumberOfVersions{1} %Poner a 2 en la versión final
\NoRearrange % Eliminar para generar el examen definitivo
\ContinuousNumbering
\IncludeFromFile{codigos_simon.tex}

\newcommand{\tab}{\hspace*{2em}}

\begin{document}

\begin{examtop}
%  \noindent Nombre:\rule{4in}{.4pt}\\
%  \noindent Apellidos:\rule{4in}{.4pt}\\
%  \noindent Grupo:    \rule{4in}{.4pt}\\
%  \noindent DNI:      \rule{4in}{.4pt}\\
    \noindent Apellidos y Nombre:\dotfill \\
    \noindent DNI: \dotfill Grupo: \dotfill \\
  \begin{center}
    \textbf{Informática --- Examen parcial de enero 2018} \\
    \textbf{Grados en Matemáticas. Grupos A, B, C, D y E} \\
    \textbf{Facultad de Ciencias Matemáticas, UCM} \\
    Tipo \Alph{version}\\
  \end{center}
\end{examtop}

\begin{exampreface}
\textbf{Instrucciones}:~\\
\begin{itemize}
\item El examen durará \textbf{3 horas}. 
\item Elegir una respuesta incorrecta en una pregunta de opción múltiple (preguntas 1--10) penalizará la nota en 0.3 puntos. Las respuestas no contestadas no penalizarán la nota. En todo caso, la puntuación mínima de la parte tipo \textit{test} será 0. 
\item No se permite la utilización de apuntes, libros o cualquier otro tipo de material en el examen. 

%Elegir una respuesta incorrecta en una pregunta de opción múltiple (preguntas 1--10) no penalizará la nota.
\end{itemize}



% Habrá XXX preguntas de respuesta múltiple, cada una de XXX puntos. Cada pregunta
% tendrá 4 opciones, de las cuales solo una es correcta. Las respuestas incorrectas
% no restan. También habrá preguntas de rellenar fragmentos de código.
% %
% Luego habrá un ejercicio de desarrollar sobre listas de 2,5 puntos.
\end{exampreface}


\begin{multiplechoice}[
title={Preguntas de opción múltiple (5 puntos)}, 
suppressprefix]
    
  \begin{question}[0.5 pt]
  Suponiendo la siguiente definición:
  \small\InsertChunk{if-elif-else_execution_order}\normalsize
  ¿Qué se imprime cuando se ejecuta la siguiente línea de código en el interprete de Python?\\
  \lstinline{admit(17, "15", False, False), admit(32, "R18", True, False)}
  % Simon:
  % To work correctly, giving the answer (True, True),
  % the first three clauses should be in the opposite order
  \choice{\lstinline{(True, True)}}
  \choice{\lstinline{(True, False)}}
  \choice{\lstinline{(False, True)}}
  \choice[!]{\lstinline{(False, False)}}
  \end{question}

  \begin{question}[0.5 pt]
  ¿Qué valor devolvería la llamada \lstinline{isPrime(51)}?
  \small\InsertChunk{badly_placed_increment}\normalsize
  % Simon:
  % To work correctly, giving the answer True
  % the indentation of the statement "i +=1" should be one level less
  \choice{\lstinline{True}}
  \choice{\lstinline{False}}
  \choice{\lstinline{'not has_divisor'}}
  \choice[!]{{Nada, el bucle no termina.}}
  \end{question}
  
  \begin{question}[0.5 pt]
  ¿Qué valor devolvería la llamada \lstinline{check("barco", 5, 'casa')}?
  \InsertChunk{and-clause_execution_order}
  % Simon:
  % The clause order is inverted w.r.t. last year's question
  % to check that the principle is understood.
  \choice{\lstinline{True}}
  \choice{\lstinline{False}}
  \choice{\lstinline{'False'}}
  \choice[!]{Se produce un error en tiempo de ejecución.}
  \end{question}
  
  \begin{question}[0.5 pt]
  ¿Qué valor devolvería la llamada \lstinline{highest_square(71)}?
  \InsertChunk{badly_placed_return}
  % Simon:
  % To work correctly, giving the answer 64,
  % the indentation of the return statement should be one level less
  \choice[!]{\lstinline{1}}
  \choice{\lstinline{64}}
  \choice{\lstinline{71}}
  \choice{\lstinline{81}}
  \end{question}

  \begin{question}[0.5 pt]
  ¿Qué valor devolvería la llamada \lstinline{number_list(3)}?
  \InsertChunk{badly_used_inverse_range}
  % Simon:
  % To work correctly, giving the answer [3, 2, 1, 3, 2, 3],
  % the first two arguments of the second range should be swapped
  \choice[!]{\lstinline{[]}}
  \choice{\lstinline{[3, 2, 1, 3, 2, 3]}}
  \choice{\lstinline{[3, 3, 2, 3, 2, 1]}}
  \choice{\lstinline{[3, 2, 1, 0, -1]}}
  \end{question}

\end{multiplechoice}

\end{document}

%%% Local Variables:
%%% mode: latex
%%% TeX-master: t
%%% End:
