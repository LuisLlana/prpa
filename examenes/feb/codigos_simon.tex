\begin{chunk}{if-elif-else_execution_order}
\begin{lstlisting}
# UK film ratings are as follows: U, PG, 12A, 12, 15, 18, R18
def admit(customer_age, film_rating, cinema_Xlicence, accompanied):
    if customer_age >= 12:
        admission = film_rating not in ["R18", "18", "15"]
    elif customer_age >= 15:
        admission = film_rating not in ["R18", "18"]
    elif customer_age >= 18:
        admission = True
        if film_rating == "R18":
            admission = cinema_Xlicence
    else:
        admission = (film_rating == "12A" and accompanied) \
            or (film_rating in ["PG", "U"]
    return admission
\end{lstlisting}
\end{chunk}

\begin{chunk}{badly_placed_increment}
\begin{lstlisting}
def isPrime(number):
    top = int(number**0.5)+1
    i = 2
    has_divisor = False
    while i < top and not has_divisor:
        if number%i == 0:
            has_divisor = True
            i += 1
    return not has_divisor
\end{lstlisting}
\end{chunk}

\begin{chunk}{and-clause_execution_order}
\begin{lstlisting}
def check (a, b, c) :
    return ( a[5] > c[0] ) and ( b%3 < 2 )
\end{lstlisting}
\end{chunk}

\begin{chunk}{badly_placed_return}
\begin{lstlisting}
def isPerfectSquare(number):
    return (int(number**0.5))**2 == number

def highest_square(top):
    highest = current = 1
    while current < top:
        if isPerfectSquare(current):
            highest = current
        current += 1
        return highest
\end{lstlisting}
\end{chunk}

\begin{chunk}{badly_used_inverse_range}
\begin{lstlisting}
def number_list(top):
    output = []
    for i in range(top):
        for j in range(i, top, -1):
            output.append(j)
    return output
\end{lstlisting}
\end{chunk}

