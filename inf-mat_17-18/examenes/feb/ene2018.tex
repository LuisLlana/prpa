\documentclass[10pt]{examdesign}
\usepackage[utf8]{inputenc}
\usepackage[spanish]{babel}
\selectlanguage{spanish}
\usepackage{listings}
\usepackage[margin=1in]{geometry}
\usepackage{enumerate}

\lstset{
  language=Python,
  frame=single,
  basicstyle=\sffamily,
  showstringspaces=false
  %basicstyle=\ttfamily,
}

\NumberOfVersions{4} %Poner a 2 en la versión final
\NoRearrange % Eliminar para generar el examen definitivo
\ContinuousNumbering
\IncludeFromFile{codigos_2018.tex}

\newcommand{\tab}{\hspace*{2em}}

\begin{document}

\begin{examtop}
%  \noindent Nombre:\rule{4in}{.4pt}\\
%  \noindent Apellidos:\rule{4in}{.4pt}\\
%  \noindent Grupo:    \rule{4in}{.4pt}\\
%  \noindent DNI:      \rule{4in}{.4pt}\\
    \noindent Apellidos y Nombre:\dotfill \\
    \noindent DNI: \dotfill Grupo: \dotfill \\
  \begin{center}
    \textbf{Informática --- Examen parcial de enero 2018} \\
    \textbf{Grados en Matemáticas. Grupos A, B, C, D y E} \\
    \textbf{Facultad de Ciencias Matemáticas, UCM} \\
    Tipo \Alph{version}\\
  \end{center}
\end{examtop}

\begin{exampreface}
\textbf{Instrucciones}:~\\
\begin{itemize}
\item El examen durará \textbf{3 horas}.
\item Elegir una respuesta incorrecta en una pregunta de opción múltiple (preguntas 1--10) penalizará la nota en 0.15 puntos. Las respuestas no contestadas no penalizarán la nota. En todo caso, la puntuación mínima de la parte tipo \textit{test} será 0.
\item No se permite la utilización de apuntes, libros o cualquier otro tipo de material en el examen.

%Elegir una respuesta incorrecta en una pregunta de opción múltiple (preguntas 1--10) no penalizará la nota.
\end{itemize}



% Habrá XXX preguntas de respuesta múltiple, cada una de XXX puntos. Cada pregunta
% tendrá 4 opciones, de las cuales solo una es correcta. Las respuestas incorrectas
% no restan. También habrá preguntas de rellenar fragmentos de código.
% %
% Luego habrá un ejercicio de desarrollar sobre listas de 2,5 puntos.
\end{exampreface}


\begin{multiplechoice}[
title={Preguntas de opción múltiple (5 puntos)},
suppressprefix]

  \begin{question}[0.5 pt]
  Suponiendo la siguiente definición:
  \small\InsertChunk{if-elif-else_execution_order}\normalsize
  ¿Qué se imprime cuando se ejecuta la siguiente línea de código en el interprete de Python?\\
  \lstinline{admit(17, "15", False, False), admit(32, "R18", True, False)}
  % Simon:
  % To work correctly, giving the answer (True, True),
  % the first three clauses should be in the opposite order
  \choice{\lstinline{(True, True)}}
  \choice{\lstinline{(True, False)}}
  \choice{\lstinline{(False, True)}}
  \choice[!]{\lstinline{(False, False)}}
  \end{question}

  \begin{question}[0.5 pt]
  ¿Qué valor devolvería la llamada \lstinline{isPrime(51)}?
  \small\InsertChunk{badly_placed_increment}\normalsize
  % Simon:
  % To work correctly, giving the answer True
  % the indentation of the statement "i +=1" should be one level less
  \choice{\lstinline{True}}
  \choice{\lstinline{False}}
  \choice{\lstinline{'not has_divisor'}}
  \choice[!]{{Nada, el bucle no termina.}}
  \end{question}

  \begin{question}[0.5 pt]
  ¿Qué valor devolvería la llamada \lstinline{check("barco", 5, 'casa')}?
  \InsertChunk{and-clause_execution_order}
  % Simon:
  % The clause order is inverted w.r.t. last year's question
  % to check that the principle is understood.
  \choice{\lstinline{True}}
  \choice{\lstinline{False}}
  \choice{\lstinline{'False'}}
  \choice[!]{Se produce un error en tiempo de ejecución.}
  \end{question}


  \begin{question}[0.5 pt]
  ¿Qué valor devolvería la llamada \lstinline{number_list(3)}?
  \InsertChunk{badly_used_inverse_range}
  % Simon:
  % To work correctly, giving the answer [3, 2, 1, 3, 2, 3],
  % the first two arguments of the second range should be swapped
  \choice[!]{\lstinline{[]}}
  \choice{\lstinline{[3, 2, 1, 3, 2, 3]}}
  \choice{\lstinline{[3, 3, 2, 3, 2, 1]}}
  \choice{\lstinline{[3, 2, 1, 0, -1]}}
  \end{question}

  \begin{question}[0.5 pt]
  ¿Qué valor devolvería la llamada \lstinline{highest_square(71)}?
  \InsertChunk{badly_placed_return}
  % Simon:
  % To work correctly, giving the answer 64,
  % the indentation of the return statement should be one level less
  \choice[!]{\lstinline{1}}
  \choice{\lstinline{64}}
  \choice{\lstinline{81}}
  \choice{Nada, el bucle no termina.}
  \end{question}

  \begin{question}[0.5 pt]
  ¿Que muestra por pantalla la llamada \lstinline{func1(3)}?
  \InsertChunk{function_calls}
  \choice{\lstinline{1,8,9}}
  \choice[!]{

         \lstinline{8}

         \lstinline{9}
  }
  \choice{

         \lstinline{1}

         \lstinline{8}

         \lstinline{9}}
  \choice{\lstinline{8,9}}
  \end{question}


  % \begin{question}[0.5 pt]
  % ¿Qué devolvería la llamada a \lstinline{find_multiples(2, [2,3,4,6,8,9,11])}?
  % \InsertChunk{bad_count_while}
  % \choice{nada, el programa tiene errores de sintaxis}
  % \choice{\lstinline{[2,4,6,8]}}
  % \choice{\lstinline{[3,9,11]}}
  % \choice[!]{Nada, el bucle no terminaría.}
  % \end{question}

  \begin{question}[0.5pt]
    ¿Qué lista genera la llamada \lstinline{create_lst()}?
    \InsertChunk{double_while}
    \choice{[[0,2], [1,3], [2,4], [3,5], [4,6], [5,7], [6,8], [7,9]]}
    \choice[!]{[[0,2], [1,3], [2,4], [3,5], [4,6], [5,7], [6,8], [7,9], [8,10]]}
    \choice{[[0,2]]}
    \choice{Ninguna de las opciones anteriores es valida}
  \end{question}

  \begin{question}
    ¿Cuál es el valor de a tras la instrucción \lstinline{a = sum_lst([1,2,3,4,5])}?
    \InsertChunk{bad_return}
      \choice{15}
      \choice{None}
      \choice{Se produce un error en tiempo de ejecución.}
      \choice{Nada, el bucle no termina.}
  \end{question}

  \begin{question}
    Cual es la salida que produce el fragmento de codigo siguiente:
    \InsertChunk{bad_print}
    Dadas las siguientes afirmaciones, ¿cuáles son ciertas?

    \begin{enumerate}[1]
    \item El código imprime \\ 0 \\ 1 \\ 2 \\ 3 \\ 4
    \item El código imprime \\ 1 \\ 2 \\ 3 \\ 4 \\ 5
    \item Se produce un error en tiempo de ejecución.
    \end{enumerate}
    \choice{Sólo la 1}\choice{Sólo la 2}\choice{La 1 y la 3}\choice{La
    2 y la 3}
  \end{question}

  \begin{question}
    ¿Cuál es la salida que produce el fragmento de código siguiente
    para la llamada \lstinline{test_if(0)}?
    \InsertChunk{if_without_else}
    \choice{Value less than or equal to zero}
   \choice{Value greater than zero}
    \choice{Value less than or equal to zero\\
      Value greater than zero}
    \choice{El código es incorrecto}
  \end{question}


\end{multiplechoice}



\begin{fillin}
[title={Rellena el código que falta (1 punto)},
suppressprefix]


  % \begin{question}[0.5 pt]
  % Rellenar los huecos para que la siguiente funcion devuelva la subcadena de \lstinline{word} que va desde la posición \lstinline{ini} y tiene \lstinline{num} caracteres.
  % \InsertChunk{fill_sub_str}
  % \end{question}

  \begin{question}[0.5 pt]
  Rellena los huecos para que la siguiente función devuelva la posición de todas las apariciones de \lstinline{word} en
\lstinline{text}. Para ello usa el método \lstinline{find} de las cadenas de caracteres:\lstinline{s.find(other, pos)} devuelve la primera aparición posterior o igual a \lstinline{pos} de \lstinline{other} en \lstinline{s}. Si \lstinline{other} no aparece devuelve -1.

  \InsertChunk{fill_all_ocurences}
  \end{question}

  % \begin{question}[0.5 pt]
  % Rellena los huecos para que la  función \lstinline{divisible_11} indique si un número es divisible por 11.

  % \InsertChunk{fill_divisible_11}
  % \end{question}

  \begin{question}[0.5 pt]
  Rellena los huecos para que la siguiente función \lstinline{exp_of} calcule el mayor \lstinline{n}
tal que existe un \lstinline{c} tal que \lstinline{num = base^n * c}
  \InsertChunk{fill_exp_of}
  \end{question}


\end{fillin}

\begin{shortanswer}
[title={Preguntas para desarrollar (4 puntos)},  suppressprefix]
  \begin{question}[2 pt]
  Como todos sabemos todo número natural $num$ mayor o igual que $1$ se puede
expresar de la siguiente forma $num = 2^a\cdot 3^b\cdot c$, donde
$a,b\geq 0$ y $c\geq 1$ no es divisible ni por $2$ ni por $3$.
Un entero $num$ es piritiguai si $num\geq 1$ y en las condiciones $a\geq b$. Una lista
de enteros será piritiguai si todos sus números son piritiguais.
Haz un programa que indique si una lista es piritiguai.
  \end{question}

  \begin{question}[2 pt]
  Dada una cadena de caracteres, queremos separar las palabras que las componen.
Consideraremos como palabra una sucesión de letras entre caracteres que no sean letras. Por ejemplo,
las palabras de \lstinline{\"En un lugar de la mancha, de cuyo nombre
  ........\"}
son: \lstinline{\"En\"}, \lstinline{\"un\"}, \lstinline{\"lugar\"},
\lstinline{\"de\"},
\lstinline{\"la\"}, \lstinline{\"mancha\"}, \lstinline{\"de\"}, \lstinline{\"cuyo\"} y \lstinline{\"nombre\"}.
Para distinguir si un carácter es una letra o un signo de puntuación o separador puedes suponer definida la siguiente función:
  \InsertChunk{def_is_letter}

Escribe una función en Python que dada una cadena de caracteres devuelva la lista de las palabras que contiene.


  \end{question}




\end{shortanswer}

\end{document}

%%% Local Variables:
%%% mode: latex
%%% TeX-master: t
%%% End:
