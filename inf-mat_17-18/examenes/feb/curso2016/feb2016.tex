\documentclass[10pt]{examdesign}
\usepackage[utf8]{inputenc}
\usepackage[spanish]{babel}
\selectlanguage{spanish}
\usepackage{listings}
\usepackage[margin=1in]{geometry}


\lstset{
  language=Python,
  frame=single,
  basicstyle=\sffamily,
  showstringspaces=false
  %basicstyle=\ttfamily,
}

\NumberOfVersions{2} %Poner a 2 en la versión final
%\NoRearrange % Eliminar para generar el examen definitivo
\ContinuousNumbering
\IncludeFromFile{codigos.tex}

\newcommand{\tab}{\hspace*{2em}}

\begin{document}

\begin{examtop}
%  \noindent Nombre:\rule{4in}{.4pt}\\
%  \noindent Apellidos:\rule{4in}{.4pt}\\
%  \noindent Grupo:    \rule{4in}{.4pt}\\
%  \noindent DNI:      \rule{4in}{.4pt}\\
    \noindent Apellidos y Nombre:\dotfill \\
    \noindent DNI: \dotfill Grupo: \dotfill \\
  \begin{center}
    \textbf{Informática --- Examen parcial de febrero 2017} \\
    \textbf{Grados en Matemáticas. Grupos A, B, C, D y E} \\
    \textbf{Facultad de Ciencias Matemáticas, UCM} \\
    Tipo \Alph{version}\\
  \end{center}
\end{examtop}

\begin{exampreface}
\textbf{Instrucciones}:~\\
\begin{itemize}
\item El examen durará \textbf{3 horas}. 
\item Elegir una respuesta incorrecta en una pregunta de opción múltiple (preguntas 1--10) penalizará la nota en 0.3 puntos. Las respuestas no contestadas no penalizarán la nota. En todo caso, la puntuación mínima de la parte tipo \textit{test} será 0. 
\item No se permite la utilización de apuntes, libros o cualquier otro tipo de material en el examen. 

%Elegir una respuesta incorrecta en una pregunta de opción múltiple (preguntas 1--10) no penalizará la nota.
\end{itemize}



% Habrá XXX preguntas de respuesta múltiple, cada una de XXX puntos. Cada pregunta
% tendrá 4 opciones, de las cuales solo una es correcta. Las respuestas incorrectas
% no restan. También habrá preguntas de rellenar fragmentos de código.
% %
% Luego habrá un ejercicio de desarrollar sobre listas de 2,5 puntos.
\end{exampreface}


\begin{multiplechoice}[
title={Preguntas de opción múltiple (5 puntos)}, 
suppressprefix]
    
%   \begin{question}
%   ¿Qué salida produciría por pantalla el siguiente código?
%   \InsertChunk{basic1}  
%   \choice[!]{\lstinline{2 4}}
%   \choice{\lstinline{a b}}
%   \choice{\lstinline{10 4}}
%   \choice{\lstinline{2.5 4}}
%   \end{question}


  \begin{question}[0.5 pt]
  ¿Qué valor devolvería la llamada \lstinline{check("barco", 5, 'casa')}?
  \InsertChunk{basic2}
  % RAFA
  % La respuesta correcta en todo caso es 1
  % \choice{La invocación a la función es errónea y no devolvería ningún
  %   valor.}
  % \choice{Depende del valor de c}  % Propuesta para aniadir

  \choice{Se produce un error en tiempo de ejecución.}
  \choice{\lstinline{True}}
  \choice[!]{\lstinline{False}}
  \choice{\lstinline{'False'}}
  \end{question}
  
  
  \begin{question}[0.5 pt]
  ¿Qué valor devolvería la llamada \lstinline{grade(2)}?
  \InsertChunk{cond1}
  \choice{\lstinline{'Suspenso'}}
  \choice{\lstinline{'Aprobado'}}
  \choice[!]{\lstinline{'Notable'}}
  \choice{\lstinline{'Sobresaliente'}}
  \end{question}
  
  \begin{question}[0.5 pt]
  ¿Qué valor devolvería la llamada \lstinline{range(7, 5, -1)}?
  \choice{\lstinline{[7, 6, 5]}}
  \choice[!]{\lstinline{[7, 6]}}
  \choice{\lstinline{[7, 5]}}
  \choice{\lstinline{[ ]}}
  \end{question}
  
  \begin{question}[0.5 pt]
  ¿Qué valor devolvería la llamada \lstinline{divisors(6)}?
  \InsertChunk{loop1}
  \choice{\lstinline{4}}
  \choice{\lstinline{2}}
  \choice{\lstinline{[6, 3, 2, 1]}}
  \choice[!]{Nada, el bucle no termina.}
  \end{question}
  
  \begin{question}[0.5 pt]
  ¿Qué valor devolvería la llamada \lstinline{create_image()}?
% RAFA. Recordar que 255 es blanco y 0 negro?
  \InsertChunk{loop2}
  \choice{Una imagen de $50$ píxeles de ancho y $100$ de alto con fondo gris y una circunferencia blanca en el centro de la imagen.}
  \choice[!]{Nada. Se produce un error del tipo {\it image index out of range.}}
  \choice{Una imagen de $100$ píxeles de ancho y $50$ de alto con fondo negro y una línea blanca vertical.}
  \choice{Nada, el bucle no termina.}
  \end{question}
  
  \begin{question}[0.5 pt]
  ¿Qué valor devolvería la llamada \lstinline{first_later([1,3,5,7], [2,4,4,7,8])}?

  \InsertChunk{cond22}
  \choice{\lstinline{None}}
  \choice{\lstinline{4}}
  \choice[!]{\lstinline{5}}
  \choice{\lstinline{Error}}

  \end{question}




  \begin{question}[0.5 pt]
  ¿Qué valor devolvería la llamada \lstinline{stellar_evolution(15.6)}?
  \InsertChunk{cond3}
  \choice{\lstinline{'supernova - neutron star / black hole'}}
  \choice{\lstinline{'none'}}
  \choice[!]{\lstinline{'blue dwarf - white dwarf - black dwarf'}}
  \choice{Fallaría al comparar números enteros y flotantes.}
  \end{question}


   \begin{question}[0.5 pt]
   ¿Qué valor devolvería la llamada \lstinline{old_car('day', 'empty', 'es')}?
   \InsertChunk{cond4}
   \choice{\lstinline{'no headlights'}}
   \choice[!]{\lstinline{'dipped headlights'}}
   \choice{\lstinline{'full beam headlights'}}
   \choice{\lstinline{None}}
   \end{question}

   
   \begin{question}[0.5 pt]
   ¿Qué devolvería la invocación a la función \lstinline{find_multiples(3,11)}?
   \InsertChunk{loop3}
   \choice{\lstinline{[3,6,9]}}
   \choice{\lstinline{[0,3,6,9]}}
   \choice[!]{Nada, el bucle no terminaría.}
   \choice{Nada, el programa tiene errores de sintaxis.}
   \end{question}

   
   \begin{question}[0.5 pt]
    Considerando el siguiente código, ¿qué mostraría por pantalla la invocación a la función \lstinline{do1(7)}?
    \InsertChunk{basic4}
    \choice{\lstinline{4.0}\\\lstinline{-4}}
    \choice[!]{\lstinline{4.0}\\\lstinline{3}}
    \choice{\lstinline{4}\\\lstinline{-4}}
    \choice{\lstinline{3}\\\lstinline{4.0}}
   \end{question}
	

%%%%%%%%%%%%%%%%%%%%%%%%%%%%%%%
%% RELLENAR CÓDIGO


\end{multiplechoice}


\pagebreak




\begin{fillin}
[title={Rellena el código que falta (1 punto)}, 
suppressprefix]

\begin{question}[0.5 pt]
  Completa el código que falta en la función 
  \lstinline{mcd} para que calcule el máximo común divisor de 
  los números $a$ y $b$ (que supondremos $\geq 0$).
  El algoritmo se basa en las siguientes premisas: \lstinline{mcd(a, b) = mcd(b, a)};
  si \lstinline{a >= b}, \lstinline{mcd(a, b) = mcd(a-b, b)}; 
  \lstinline{mcd(a, 0) = a} \\~\\
%  {\sffamily
%    def len\_base10(n):\\
%    \tab n = abs(n)\\
%    \tab c = 0\\
%    \tab while \blank{n $>$
%      0\phantom{mmmmmmmmmmmmmm}}:\\ %Para dejar más hueco
%    \tab\tab n = \blank{n // 10\phantom{mmmmmmmmmmmmmm}}\\
%    \tab\tab c = c + 1\\
%    \tab return c\\
%  }
  
    {\sffamily
    def mcd(a, b):\\
    \tab while b != 0:\\
        \tab\tab \blank{if a $<$ b\phantom{mmmmmmmmmmmm}}:\\
            \tab\tab\tab  \blank{a, b = b , a \phantom{mmmmmmmmm}}\\
        \tab\tab \blank{a = a - b\phantom{mmmmmm}}\\
    \tab return a 
            
  }
\end{question}

% \begin{question}[0.5 pt]
%   Completa la función \lstinline{quadratic} para calcular las dos
%   soluciones a para
%   la ecuación $ax^2 + bx + c = 0$, considerando que $a > 0$ y $b^2 -4ac> 0 $.\\~\\
%   {\sffamily
%     import math\\
%     ~\\
%     def quadratic(a, b, c):\\
%     \tab disc = b*b - 4*a*c\\
%     \tab sol1 = \blank{(-b + math.sqrt(disc)) / (2*a)\phantom{mmmmmmmmmmmmmm}}\\
%     \tab sol2 = \blank{(-b - math.sqrt(disc)) / (2*a)\phantom{mmmmmmmmmmmmmm}}\\
%     \tab return \blank{sol1, sol2\phantom{mmmmmmmmmmmmmm}}\\
%   }
% \end{question}

\begin{question}[0.5 pt]
Completa la función \lstinline{reverse\_num} que calcula el \emph{reverso}
de un número entero. El resultado es otro número entero. Por ejemplo, el reverso de 1234
es 4321 y el reverso de 12000 es 21. \\
~\\
{\sffamily
def reverse\_num(number): \\
   \tab  reverse = 0\\
   \tab while number != 0:\\
   \tab\tab     reverse = \blank{10* reverse + (number \% 10)\phantom{mmmmmmmm}}\\
   \tab\tab     number = \blank{number / 10\phantom{mmmmmmmm}}\\
   \tab  return reverse \\
}


%{\sffamily
%def binary\_search(e, l):\\
%\tab i = 0\\
%\tab j = len(l) - 1\\
%\tab result = False\\
%~\\ 
%\tab while \blank{(i $<=$ j) and (not result)\phantom{mmmmmmmmmmmmmm}}:\\
%\tab \tab med = \blank{(i + j) // 2\phantom{mmmmmmmmmmmmmm}}\\
%\tab \tab if l[med] == e:\\
%\tab \tab \tab result = True\\
%\tab \tab elif l[med] $<$ e:\\
%\tab \tab \tab \blank{i = med + 1\phantom{mmmmmmmmmmmmmm}}\\
%\tab \tab elif l[med] $>$ e:\\
%\tab \tab \tab \blank{j = med - 1\phantom{mmmmmmmmmmmmmm}}\\
%~\\            
%\tab return result\\
%}
\end{question}


\end{fillin}

\begin{shortanswer}
[title={Preguntas para desarrollar (4 puntos)},  suppressprefix]

%   \begin{question}
%     \begin{itemize}
%     \item Dado un espacio vectorial $\mathbf{E}$ sobre un cuerpo $K$ y
%       definido el producto escalar para cada uno de sus vectores como
%       \begin{displaymath}
%         \mathbf{v \cdot w}=v_{1}\cdot w_{1}+ \dots + v_{n}w_{n}
%       \end{displaymath}
%       normalizar un vector consiste en dividirlo por su norma, siendo
%       esta
%       \begin{displaymath}
%         |\mathbf{v}| = \sqrt{v_{1}\cdot v_{1}+ \dots + v_{n}v_{n}}
%       \end{displaymath}
% 
% 
%       \emph{Representando} los vectores como listas de coordenadas respecto a
%       una base, programar una función \lstinline{normalize(v)} al que
%       pasado un vector \emph{no nulo} de dimensión \textbf{n},
%       devuelva las coordenadas de su versión normalizada. (Se acepta
%       que cambie el argumento)
%       \item Responde razonadamente \textbf{Que consecuencias tiene que la representación de los
%       reales no sea precisa? }
%   \end{itemize}
% 
% \begin{answer}
%   ~
%   \InsertChunk{problem2}
%   \end{answer}  
% 
%   \end{question}

  

\begin{question}[2 pt]
  Una imagen es \emph{oscura} cuando una proporción significativa de sus
  píxeles son oscuros. 
  Si la imagen es en blanco y negro, el píxel es
  \emph{oscuro} cuando tenga un valor cercano a 0. En este ejercicio se
  pide hacer una función, que se llamará \lstinline{is_dark} 
  que indique si una imagen es oscura o no.
  La función tendrá un primer parámetro \lstinline{image} que hará
  referencia a un objeto imagen. Como ``ser cercano a cero''
  puede ser algo subjetivo, la función tendrá un segundo parámeto
  \lstinline{level} que indicará valor de forma que todos los
  inferiores a él se considerarán oscuros. Por último debemos indicar
  cual es la proporción de valores oscuros, para ello la función tendrá
  un tercer parámetro \lstinline{ratio} que indicará que proporción de
  valores oscuros debe tener la imagen para considerarla oscura. El
  parámetro \lstinline{ratio} es un valor en el intervalo $(0,1)$. Por
  ejemplo, si consideramos la imagen \lstinline{img} con los siguientes valores:
  \begin{center}
    \begin{tabular}{*{6}{r}}
      0 &   1 &   2 &   3 &   4 &   5 \\
      6 &   7 &   8 &   9 &  10 &  11 \\
      12 &  13 &  14 &  15 &  16 &  17 \\
      18 &  19 &  20 &  21 &  22 &  23 \\
      24 &  25 &  26 &  27 &  28 &  29 \\
      30 &  31 &  32 &  33 &  34 &  35 \\
    \end{tabular}
  \end{center}
  Tendremos las siguientes respuestas:\\
  \lstinline{is\_dark(img, 27, .5)} $\to$  \lstinline{True}\\
  \lstinline{is\_dark(img, 10, .5)} $\to$ \lstinline{False} \\
  \begin{answer}
  ~
  
    \InsertChunk{isDark}
  \end{answer}
\end{question}
  
\begin{question}[2 pt]
  Escribe una función \lstinline{communitary(lst)} para comprobar si una lista de 
  números enteros es \emph{comunitaria}. Diremos que una lista es 
  comunitaria si existe un divisor $d > 1$ común a todos
  los enteros de la lista. Ejemplos:\\[-.2cm]
   ~\\
    \texttt{
    communitary([2, 4, 6, 8]) $\to$ True \\
    communitary([2, 4, 6, 8, 3]) $\to$ False
    }


  \begin{answer}
  ~
  \InsertChunk{communitary}
  \end{answer}  
  \end{question}  
\end{shortanswer}


\end{document}

%%% Local Variables:
%%% mode: latex
%%% TeX-master: t
%%% End:
