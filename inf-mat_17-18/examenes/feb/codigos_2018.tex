\begin{chunk}{if-elif-else_execution_order}
\begin{lstlisting}
# UK film ratings are as follows: U, PG, 12A, 12, 15, 18, R18
def admit(customer_age, film_rating, cinema_Xlicence, accompanied):
    if customer_age >= 12:
        admission = film_rating not in ["R18", "18", "15"]
    elif customer_age >= 15:
        admission = film_rating not in ["R18", "18"]
    elif customer_age >= 18:
        admission = True
        if film_rating == "R18":
            admission = cinema_Xlicence
    else:
        admission = (film_rating == "12A" and accompanied) \
            or (film_rating in ["PG", "U"]
    return admission
\end{lstlisting}
\end{chunk}

\begin{chunk}{badly_placed_increment}
\begin{lstlisting}
def isPrime(number):
    top = int(number**0.5)+1
    i = 2
    has_divisor = False
    while i < top and not has_divisor:
        if number%i == 0:
            has_divisor = True
            i += 1
    return not has_divisor
\end{lstlisting}
\end{chunk}

\begin{chunk}{and-clause_execution_order}
\begin{lstlisting}
def check (a, b, c) :
    return ( a[5] > c[0] ) and ( b%3 < 2 )
\end{lstlisting}
\end{chunk}

\begin{chunk}{badly_placed_return}
\begin{lstlisting}
def isPerfectSquare(number):
    return (int(number**0.5))**2 == number

def highest_square(top):
    highest = current = 1
    while current < top:
        if isPerfectSquare(current):
            highest = current
        current += 1
        return highest
\end{lstlisting}
\end{chunk}

\begin{chunk}{badly_used_inverse_range}
\begin{lstlisting}
def number_list(top):
    output = []
    for i in range(top):
        for j in range(i, top, -1):
            output.append(j)
    return output
\end{lstlisting}
\end{chunk}

\begin{chunk}{function_calls}
\begin{lstlisting}
def func1(x):
    a = func2(x//2)
    b = func3(x+1)
    print(a + b)

def func2(x):
    return x**2
    print(x**2)

def func3(y):
    if y < 4:
       print(y**2)
    else:
       print(y*2)
    return 2*y
\end{lstlisting}
\end{chunk}

\begin{chunk}{bad_count_while}
\begin{lstlisting}
def find_multiples(x, l):
    i = 0
    multiples = []
    while i < len(l):
        if l[i] % x == 0:
            multiples.append(l[i])
    return multiples
\end{lstlisting}
\end{chunk}

\begin{chunk}{fill_sub_str}
\begin{lstlisting}
def substr(word, ini, num):
    i = 0
    sub = ""
    while _______________________:
        _______________
        i += 1
    return sub
\end{lstlisting}
\end{chunk}

\begin{chunk}{fill_all_ocurences}
\begin{lstlisting}
def all_occurrences(text, word):
    result = []
    last = __________
    while last != -1:
        result.append(last)
        last = ____________
    return result
\end{lstlisting}
\end{chunk}

\begin{chunk}{fill_divisible_11}
\begin{lstlisting}
def sum_par_impar(n):
    pos_par = ______
    pares = ____
    impares = _____

    while  n!=0:
        digit = __________
        if pos_par :
            pares = pares + digit
        else:
            impares = impares + digit
        n = ______
        pos_par = __________
    return (pares, impares)


def divisible_11(n):
    while ____:
        pares, impares = sum_par_impar(n)
        if ______:
            n = pares - impares
        else:
            n = impares - pares
    return n==0 or n==11
\end{lstlisting}
\end{chunk}

\begin{chunk}{fill_exp_of}
\begin{lstlisting}
def exp_of(base, num):
    exp = 0
    while _____:
        num = _______
        exp = _______
    return exp
\end{lstlisting}
\end{chunk}

\begin{chunk}{def_is_letter}
\begin{lstlisting}
import string

def is_letter(c):
    return c not in string.whitespace and c not in string.punctuation

\end{lstlisting}
\end{chunk}

\begin{chunk}{if_without_else}
  \begin{lstlisting}
def test_if(x):
  if x <= 0:
     print("Value less than or equal to zero")
     x += 1
  if x >= 0:
     print("Value greater than zero")
  \end{lstlisting}
\end{chunk}


\begin{chunk}{sum_list}
  \begin{lstlisting}
list = [1,2,3,4,5]
sum = 10
while i < len(list):
   sum += list[i]
   i +=1
  \end{lstlisting}
\end{chunk}


\begin{chunk}{bad_print}
  \begin{lstlisting}
list = [1,2,3,4,5]
position = 0
i = 0
while i < len(list):
   print(list[position])
   position += 1
  \end{lstlisting}
\end{chunk}

\begin{chunk}{bad_return}
  \begin{lstlisting}
def sum_lst(lst):
   sum = 0
   while i < len(lst):
      sum += lst[i]
      i +=1
   print(sum)

a = sum_lst([1,2,3,4,5])
  \end{lstlisting}
\end{chunk}

\begin{chunk}{double_while}
   \begin{lstlisting}
def create_lst():
   i = 0
   j = 0
   lst = []
   while (i <= 10):
      while (j <= 10):
         if (j == i+2):
            lst.append([i,j])
         j = j+1
      i = i+1
   return lst
  \end{lstlisting}
\end{chunk}
